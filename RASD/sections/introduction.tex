\section{Introduction}\label{sec:intro}

General introduction. \todo{This is a todo}Text.

General introduction. \warning{This is a warning}Text.

General introduction. \request{This is a request}Text.

General introduction. \info{This is an info}Text.

\subsection{Purpose}
This document is the Requirement Analysis and Specification Document for the Customers Line-Up system.
The purpose of this document is to describe the system focusing on scenarios, use cases, requirements and 
specifications, analyzing what the software will do, how it will be used and the constraints under which it
will operate. This document is intended both for users and developers.

\subsection{Scope}
%\info{here we include an analysis of the world and of the shared phenomena}
In recent times, it has become clear that avoiding gathering of groups of people in small spaces is critical
for the safety of the population. During lockdowns some places become more crowded than others. 
This reason lead to the development of various solutions, all resulting roughly in restricting the capacity
of a building, without having a real knowledge of the number of people inside (as ofter required by the law).

A definitive and well-articulated system is still missing, especially considering that most people aren't much
tech-savvy, and especially elderly aren't familiar with the Internet.
According to recent statistics\footnote[1]{\url{https://www.istat.it/it/archivio/216672}}, in Italy over 30\% of the population has
never used the Internet, in particular only the 10\% of 75+ years old regularly use it; on the other hand many of
those able to surf the web do it through a smartphone and lack digital skills.

The proposed system will focus on solving the problem posed by queues in front of groceries stores but it could
be generalized for queues handling in multiple scenarios.

\emph{Customers Line-Up} (CLup) is a system that allows supermarket managers to regulate the influx of people
inside physical stores and giving customers a way to drastically reduce (or remove completely) the time spent
waiting in a queue. The idea of CLup is being more akin to an open-source framework that can be adoped and improved, rather
than it being a closed-source product.

This tool reaches the goal by offering a number of functionalities, including:
\begin{itemize}
    \item access to the service via mobile app or website
    \item physical alternatives for people that do not have Internet access
    \item monitor and dynamically restrict the amount of people allowed in a store
    \item book a visit, notifying customers of any change in the schedule
    \item restrict the store selection by using filters 
    \item suggest alternative stores and/or time frames
    \item track the time spent in the store by customers to provide better estimate of waiting times
\end{itemize}

\subsubsection{World Phenomena}
\begin{enumerate}[label={[WP\arabic*]}]
    \item new law on the limit on the maximum number of people in a store
    \item User reaches to the store
    \item User enters the store
    \item User exits the store
    \item User wants to shop
    \item User can't reach the store in time
    \item new store is built
    \item store is closed
    \item maximum number of people in a store is reached
\end{enumerate}

\subsubsection{Shared Phenomena}
\begin{enumerate}[label={[SP\arabic*]}]
    \item User makes a reservation
    \item User joins Queue
    \item User cancels reservation
    \item User leaves the Queue
    \item System cancels reservation
    \item System kicks User out of Queue
    \item User scans the QR code
    \item The system is notified that an User has finished shopping
    \item The system let an allowed User enter the store 
    \item Manager adds a store
    \item Manager removes a store
    \item Manager changes the maximum number of people allowed in a store
    \item Manager changes time and size of reservation slots
\end{enumerate}

\subsubsection{Current System}
While there are already existing similar services, some are usually independent from store chains and
therefore have limited functionalities.
Notable examples are:
\begin{itemize}
    \item \href{https://www.ufirst.com}{\emph{Ufirst}} allows customers to virtually get a number
        and get notified when it's their turn, offers an app for store managers and a tablet app for totems
        where a customer without a smartphone can get a ticket. Stores are required to register to make use
        of the service.
    \item \href{https://play.google.com/store/apps/details?id=com.codaliscia}{\emph{FilaIndiana}} allows customer to
        know how many people are in queue in front of a store by using GPS data from its userbase. Doesn't allow to
        book because it's indipendent from store chains. It's similar to what Google Maps offers with location-based
        crowd prediction.
    \item \href{https://play.google.com/store/apps/details?id=it.anybot.quandospesa}{\emph{QuandoSpesa}} works similarly
        to \emph{FilaIndiana}, but allows users to select a certain time slot that is guaranteed to be the less crowded
        for all the app users.
    
\end{itemize} 
CLup is a service that supermarket chains can implement alongside their existing services. The system is as independent as possible from existing infrastructures, and it can be used with minimal setup.

\subsubsection{Goals}
\begin{enumerate}[label={[G\arabic*]}]
    \item allow users to avoid crowds inside and outside the store
    \begin{enumerate}[label={[G1.\arabic*]}]
        \item allow users to join a virtual queue
        \item allow users to book a visit at a specific time
    \end{enumerate}
    \item allow users with special needs to have priority over regular users by providing a dedicated   queue
    \item allow users to save time by booking a reserved slot to do the groceries
    \item allow managers to have control over the store by setting the maximum number of people or allocating slots for reservations, people in need, and queue
    \item allow managers to monitor the current and past status of the store (for example, number of people in the store, fullness of the queues and slots) by tracking the enter/exit time of users
\end{enumerate}

\subsection{Definitions, Acronyms, Abbreviations}

\subsubsection{Definitions}
\begin{itemize}
    \item \emph{User} (also \emph{Customer} or \emph{Visitor}): A person that uses the system to shop at a store.
    \item \emph{Registered User}: A User that has registered an Account within the System.
    \item \emph{System Manager}: A stakeholder (owner, employee, manager etc.) of the Store chain that can tweak the parameters of the System and access informations and statistics.
    \item \emph{Account}: A reference to a specific User in the System, that allows to track the User across multiple visits.
    \item \emph{Reservation} (or \emph{Booking}): Arrangement made between a User and the System in which the System shall grant the User access to Store at the arranged time.
    \item \emph{Visit}: The time frame in which the User enters the store, shops and exits.
    \item \emph{Time slot}: The time at which a Customer with a Reservation is expected arrive at the store.
    \item \emph{Store}: Any physical location (e.g.: building) where it is possible to utilize the System.
    \item \emph{Totem}: A physical device with a touchscreen display and an attached printer that allows Customers to join the Virtual Queue.
    \item \emph{Virtual Queue}: the virtual equivalent of a physical queue in front of the store, regulating the access of people by ordering them.
    \item \emph{Web App}: A web application, consisting of a back-end and a front-end accessible from a web browser.
    \item \emph{Line}: Synonim for \emph{queue}.
\end{itemize}

\subsubsection{Acronyms}
\begin{itemize}
    \item RASD: Requirement Analysis and Specification Document
    \item API: Application Programming Interface
    \item CLup: Customer Line-up
    \item REST: REpresentational State Transfers
\end{itemize}


\subsubsection{Abbreviations}
\begin{itemize}
    \item {[Gn]}: n-goal.
    \item {[Dn]}: n-domain assumption.
    \item {[Rn]}: n-functional requirement.
    \item {[WPn]}: n-world phenomenon.
    \item {[SPn]}: n-shared phenomenon.
\end{itemize}

\subsection{Revision History}
\subsection{Reference History}
\begin{itemize}
    \item Problem Specification Document: ``Assignment AY 2020-21.pdf''
    \item \url{https://standards.ieee.org/standard/29148-2018.html}
\end{itemize}
\subsection{Document Structure}

The document is structured in three sections:

\begin{itemize}
    \item In the \emph{Introduction} we provide a brief explanation of the problem, what the proposed solution consist of and how it differs from existing systems, reference for used resources, language and revisions.
    \item In the second section, we provide an overall description of the product and its functions, with the help
    of scenarios to illustrate various situations. We also analyze the possible userbase of the service and then base
    assumptions and constraint upon them.
    \item In the third section we explain the more technical details by analyzing the software, hardware and user interfaces
    with the help of UML diagrams (sequence, activity) and mockups.
    We explore also aspects related to reliability, availability, security, maintainability and portability.
    \item In the fourth section we analyze formally the proposed models with the help of the Alloy Tool.
\end{itemize}
