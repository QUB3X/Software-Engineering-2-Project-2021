\section{Introduction}\label{sec:intro}

General introduction. \todo{This is a todo}Text.

General introduction. \warning{This is a warning}Text.

General introduction. \request{This is a request}Text.

General introduction. \info{This is an info}Text.

\subsection{Purpose}
This document is the Requirement Analysis and Specification Document for the Customers Line-Up system.
The purpose of this document is to describe the system focusing on scenarios, use cases, requirements and specifications,
analyzing what the software will do, how it will be used and the constraints under which it will operate.
This document is intended both for users and developers.

\subsection{Scope}
\info{here we include an analysis of the world and of the shared phenomena}

\emph{Customers Line-Up} (CLup) is a tool that allows supermarket managers to regulate the influx of people inside
physical stores and aims reduce the time spent by customers waiting in line, especially in emergency lockdown scenarios.
The system allows users to 
This tool reaches the goal by offering a number of functionalities, including:\todo{revise/add more}
\begin{itemize}
    \item access to the service via mobile app or website
    \item physical alternatives for people that do not have Internet access
    \item monitor the amount of people in a store
    \item book a visit, notifying customers of any change in the schedule
    \item suggest alternate stores and/or time frames
    \item track the time spent by customers to estimate waiting times
\end{itemize}

\subsubsection{Current System}
While there are already existing similar services, they are usually indipendent from store chains and
therefore have limited functionalities. CLup is a service that supermarket chains can implement alongside
their existing services. The system is as indipendent as possible from existing infrastructures, and
it can be used with minimal setup.

\subsubsection{Goals}
\begin{enumerate}
    % Users
    \item Allow a User to sign up for an Account after providing a mobile phone number.
    \item Allow a User to book a visit to a specific store.
        \begin{enumerate}
            \item Allow a User to book a visit via Mobile App.
            \item Allow a User to book a visit via Website.
            \item Allow a User to book a visit in a specified time.
            \item Allow a User to book a visit as soon as possible.
        \end{enumerate}
    \item Allow a User to find Stores nearby their current location.
    \item Allow a User to find Stores nearby a specified location.
    \item Allow a User to preview an estimate of the queue time.
    \item Allow a User to cancel their reservation.
    \item Allow a User to retrieve a scannable QR Code/Barcode that they must present in order to be granted access to a store.
    \item Allow a User without Internet access to retrive a ticket from a physical location that counts as a reservation for a certain time.
    \item Allow a User to book for someone else.
    \item Allow a User to link their Account to the Store Loyalty Program.
        \begin{enumerate}
            \item Users do not have an Account cannot are not entitled to this feature.
        \end{enumerate}
    % Store
    \item The System notifies the Users affected by delay.
    \item The System shall postpone Users visits in case of a delay.
    \item The System shall not \request{Are we sure about this?}anticipate User visits when a User delete their reservation.
    \item The System must enforce the limits on the allowed number of concurrent Customers inside a store.
        \begin{enumerate}
            \item There can be less Customers than the limit.
            \item There cannot be more Customers than the limit.
            \item The queue is updated each time a Customer exits or enters the store.
        \end{enumerate}
    \item The System shall not admit Users that arrive earlier, even if the current number of Customers isn't maximum.
    \item The System shall grant a User access only after the User's time of reservation.
    \item The System shall invalidate a User's reservation if they do not show up during a certain time interval.
    \item The System shall reserve a certain number of the allowed quote of customers for a special category of Users.
        \begin{enumerate}
            \item The system shall grant access to Users without a reservation that show up at the store and are pregnant women, elderly or with disabilities.
        \end{enumerate}
    % System Managers
    \item Allow System Managers to set a limit to the people allowed into the store at a time.
    \item Allow System Managers to not provide the physical ticket option.
    \item Allow System Managers to enable the link Account to Loyalty Program feature.
\end{enumerate}

\subsection{Definitions, Acronyms, Abbreviations}

\subsubsection{Definitions}
\begin{itemize}
    \item \emph{Customer} (also \emph{User} or \emph{Visitor}): A person that intends to shop at a store.
    \item \emph{Registered User}: A User that has registered an Account withing the System.
    \item \emph{System Manager}: A stakeholder (owner, employee, manager etc.) of the Store chain that can tweak the parameters of the System and access informations and statistics.
    \item \emph{Account}: A reference to a specific User in the System, that allows to track the User across multiple visits.
    \item \emph{Reservation} (or \emph{Booking}): Arrangement made between a User and the System in which the System shall grant the User access to Store at the arranged time.
    \item \emph{Visit}: The time frame in which the User enters the store, shops and exits.
    \item \emph{Time slot}: The time at which a Customer with a Reservation is expected arrive at the store.
    \item \emph{Store}: Any physical location (e.g.: building) where it is possible to utilize the System.
\end{itemize}

\subsubsection{Acronyms}
\begin{itemize}
    \item RASD: Requirement Analysis and Specification Document.
    \item API: Application Programming Interface
    \item CLup: Customer Line-up
    \item REST: REpresentational State Transfer
\end{itemize}


\subsubsection{Abbreviations}
\begin{itemize}
    \item {[Gn]}: n-goal.
    \item {[Dn]}: n-domain assumption.
    \item {[Rn]}: n-functional requirement.
\end{itemize}

\subsection{Revision History}
\subsection{Reference History}
\begin{itemize}
    \item Problem Specification Document: \todo{Should we upload it?}``Assignment AY 2020-21.pdf''
    \item https://standards.ieee.org/standard/29148-2018.html
\end{itemize}
\subsection{Document Structure}
