\section{Overall Description}\label{sec:overall_desc}

\subsection{Product Perspective}

here we include scenarios and further details on the shared
phenomena and a domain model (class diagrams and state charts)

Customers Line-Up is developed for both shop managers and customers.
The intent is to provide functionalities adding value to the interactions between the two.
Managers will be able to understand more about their customers through insightful analytics and will avoid big crowds inside and outside their stores.
Customers will have an easy way of avoiding lines by booking a visit to stores, and will be aided in their selection of the best place and time.
\warning{va qua?}The system will have two modes, one for customers who have a registered account and one for customer who don't.
Customer who do not have an account will not be tracked and will not be able to receive advice.

The system will be developed from scratch, giving great flexibility.
The privacy of the customers will be guaranteed according to the latest privacy related norms.

\subsubsection{Scenarios}
    \begin{enumerate}[label=\Alph*.]
        \item \warning{is the mobile app a design choice?} \textbf{Customer with Mobile App arrives in time}\\
            Ian wants to buy groceries to make a cake. Ian uses CLup to get a ticket for the supermarket with the shortest queue in his area.
            The app provides Ian with an estimate on the travel time (by car or by foot) and the time of the reserved slot. 
            Ian arrives at the supermarket in the correct time-slot, scans the generated \warning{design choice?}QR code and is granted access 
            the store. Once he pays
            his groceries he scans again his QR code, so that he can increase the points associated with his account in 
            this chain.

        \item \textbf{Customer with no knowledge on the booking system}
            Pino is an elderly man. Pino knows nothing about Smartphones or Computers. Pino needs to buy a cake for his
            nephew's birthday party. Pino decides to go to his local supermarket. At arrival Pino sees that the door for the market
            is not opening. He soon realises that on the door there are things written, and in particular an arrow pointing to a machine. 
            As soon as he approaches the machine, the machine activates itself an starts speaking with a reassuring voice. 
            The machine allows Pino to book a reservation to enter the store and instructs Pino on how to do so. 
            As soon as the time is up, Pino places his ticket onto the reader beside the door of the store, and he is able to enter. 

        \item \textbf{Customer cancels the reservation}\\
            Luigi, after booking a visit to the store, remembers that he had a visit to the dentist at the same time.
            Since Luigi cares about others, he cancels his reservation, freeing up a time-slot to be used by other customers.

        \item \textbf{Customer is unable to provide their code}\\
            Andrea books visit and reaches the store in time.
            However he is distractive and forgot to charge his phone, which turns off as he pulls it out of his pocket in order to show his code.
            \request{Come si gestisce? Nuovo o logga dal totem (ma poi si complica troppo il tutto)?}Andrea goes at the totem, makes a new reservation, 
            and recieves a new code and a new time-slot.
    \end{enumerate}






\subsection{Product Functions}\todo{fare dettagliato e numerare}
here we include the most important requirements

\begin{itemize}
    \item Manager:
    \begin{itemize}
        \item monitor the current status of all stores
        \item obtain information on the behavior of customers
    \end{itemize}
    \item Customer:
    \begin{itemize}
        \item account:
        \begin{itemize}
            \item register a new account
            \item show the reservation history
            \item obtain information related to the account, namely the average stay and the preferred stores
        \end{itemize}
        \item booking:
        \begin{itemize}
            \item book a visit to the store
            \item give advice regarding when and where to book
            \item send notifications regarding the status of the reservation, its delay, or its deletion
            \item cancel a reservation
        \end{itemize}
        \item view stores nearby and their availability
    \end{itemize}
\end{itemize}


\subsection{User Characteristics}
Customers Line-Up is mainly aimed at essential and widely used services.
Because of this its audience will be wide and diversified, and the system will be easy to use and accessible in several of ways, accounting in particular for people with disabilities or people who are not familiar with technology.
On one side of the system there is the system manager (single or multiple), who will monitor how the system is used and obtain useful information.
They are usually already familiar with othe customer relationship managers and already know what to expext from a control panel.
On the other side there is the customer, who uses the system in order to avoid boring lines and to prevent contact with others.
The main categories of customer are:
\begin{itemize}
    \item \textbf{Tech-friendly}\\
        People who are familiar with modern technologies. They find it easy to navigate the menus of a complex application.
        They are able to use the system in an autonomous way and are the ones who will benefit the most from the more complex and advanced features.
    \item \textbf{Tech-unfriendly}\\
        People who are not familiar with modern technologies. They have problems navigating complex application, and are more accustomed to talking to humans.
        They might need aid using the system or misuse the system. They benefit from a system designed around clarity and simplicity, or from different, easier ways of using the system.
    \item \textbf{People with disabilities}\\
        People with conditions preventing the use of a normal system, like lack of sight or earing.
\end{itemize}
The objective of Customers Line-Up is to be as inclusive as possible, providing utilities targeted at all possible users.

\subsection{Assumptions, Dependencies, and Constraints}
here we include domain assumptions 