% !TeX root = ../dd.tex

\section{Requirements Traceability}
% Explain how the requirements you have defined in the RASD
% map to the design elements that you have defined in this document.

In this section the requirements specified in the RASD are mapped to the components defined in this document. We'll only consider top-level components since they've already been explained. Frontend components are omitted for clarity.

\def\arraystretch{1.5}
\begin{longtable}{p{.05\textwidth} p{.60\textwidth} p{.30\textwidth}}
    \textbf{Req.} & \textbf{Description} & \textbf{Components}\\
    \hline
    % Users
    R1 & Allow a User to sign up for an Account after providing a mobile phone number. & \textbf{Account Manager}\\
    R2 & Allow a Registered User to find Stores nearby a specified location. & \textbf{Store Search}, \textbf{Maps API}\\
    R3 & Allow a Registered User to filter out stores based on available timeframes, days and distance. & \textbf{Store Search}\\
    R4 & Allow a Registered User to get in the virtual line at a specified store. & \textbf{Ticket Manager}, \textbf{Queue Manager}\\
    R5 & Allow a Totem User to get in the virtual line of the store where the totem is installed. & \textbf{Ticket Manager}, \textbf{Queue Manager}\\
    R6 & Allow a Registered User to preview an estimate of the queue time. & \textbf{Ticket Manager}, \textbf{Queue Manager}\\
    R7 & Allow a Registered User to book one visit to a specific store. & \textbf{Ticket Manager}, \textbf{Reservation Manager}\\
    R8 & Allow a Registered User to cancel their reservation. & \textbf{Ticket Manager}, \textbf{Reservation Manager}\\
    R9 & Allow a Registered User to leave the virtual queue. & \textbf{Ticket Manager},\textbf{Queue Manager}\\
    R10 & Allow a Registered User and a Totem User to retrieve a scannable QR Code/Barcode that they must present in order to be granted access to a store. & \textbf{Ticket Manager}, \textbf{Queue Manager}, \textbf{Reservation Manager}\\

    % Store
    R11 & The System notifies the Users affected by delay. & \textbf{Queue Manager}, \textbf{Reservation Manager}, \textbf{Notification API}\\
    R12 & The System cancels User reservations in case of a major delay. & \textbf{Queue Manager}, \textbf{Reservation Manager}, \textbf{Notification API}\\
    R13 & The System enforces the limits on the allowed number of concurrent Customers inside a store by restricting the access at the entry points (for example, automatic doors or turnstile). & \textbf{Ticket Manager}\\
    R14 & The System grants a User with a reservation access only within a short time (set by the manager) after the User's time of reservation. & \textbf{Ticket Manager}\\
    R15 & Allow System Managers to set the division of the maximum number of people allowed between the normal queue, the priority queue for people with special needs and the book a visit slot capacity. & \textbf{Admin Services}, \textbf{Reservation Manager}, \textbf{Queue Manager}\\
    R16 & The System calculates the average shopping time by recording every time a user enters and exits the store. & \textbf{Store API Adaptor}, \textbf{Ticket Manager}\\
    
    % System Managers
    R17 & Allow System Managers to set a limit to the people allowed into the store at a time. & \textbf{Admin Services}\\
    R18 & Allow System Managers to choose the frequency and size of the time slots. & \textbf{Admin Services}, \textbf{Reservation Manager}\\
    R19 & Allow System Managers to know the average time spent in the store. &  \textbf{Admin Services}\\
    R20 & Allow System Managers to know the current and past number of people in the store. & \textbf{Admin Services}\\
    R21 & Allow System Managers to check the current status of the queue and of the time slots. & \textbf{Admin Services}, \textbf{Reservation Manager}, \textbf{Queue Manager}\\

    \caption{Your caption here} % needs to go inside longtable environment
    \label{tab:myfirstlongtable}
\end{longtable}
    