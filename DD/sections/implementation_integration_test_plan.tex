% !TeX root = ../dd.tex

\section{Implemention, Integration and Test Plan}
% Identify here the order in which you plan
% to implement the subcomponents of your system and the order in which you plan % to integrate such subcomponents and test the integration.

\subsection{Overview}

The application is composed of three decoupled layers (\emph{client}, \emph{business}, and \emph{data}) which can be developed and unit tested independendly, and integration tested at the end.

\paragraph{Front-end components} They consist of the mobile and web application, that have been presented in Chapter \ref{ch:2}. They consist mostly of presentational components that belong to the client layer. Since both applications rely on a REST API, and should likely reuse portions of the codebase, they can be easily unit tested by mocking of the REST API.

\paragraph{Back-end components} They're components that resides in the server, from both the business and the data layer.

\paragraph{External components} They consist of the \emph{Maps API}, the \emph{SMS API}, and the \emph{Notification API}. Since they're provided by third parties, they're supposed to be reliable and conform to their specifications.

\subsection{Feature identification}

To better plan the testing each component will require, it's useful to visualize them in a table (Table \ref{tab:features}) where each components is associated with its difficulty of implementation and its importance for the system.

% TODO TABLE
\begin{table}[h]
    \centering
    \begin{tabular}[h]{|l|l|l|}
        \hline
        \textbf{Feature} & \textbf{Importance} & \textbf{Difficulty} \\\hline
        User login & High & Medium \\
        Join queue & High & Medium \\
        Reserve timeslot & High & High \\
        Search store & Medium & Medium \\
        View store details & Medium & Low \\
        Notify customers & High & Medium \\
        Adjust store parameters (managers) & High & Medium \\
        Add/Remove/Edit stores (managers) & Medium & Low \\
        View statistics (managers) & Low & Low \\
        \hline
    \end{tabular}
    \caption{Importance and difficulty of required features}
    \label{tab:features}
\end{table}

\subsection{Approach}

All components will be implemented and tested with a \emph{bottom-up} approach, in order to reduce the overhead that would have derived from a \emph{top-down} one.

Components from the same subsystem (for example, the backend) can be implemented, unit-tested and integration-tested without a real need of components from another subsystem. This allows developers to develop in parallel the client and the backend, thus speeding up the development process.

Finally, after the final integration testing is complete, it's important to verify the adherence to the specified requirements.


